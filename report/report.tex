\documentclass[11pt, a4paper]{article}

% Packages
\usepackage[francais]{babel}
\usepackage[T1]{fontenc}
\usepackage[utf8]{inputenc}

\usepackage[left=2cm, right=2cm, top=2cm, bottom=2cm]{geometry}
\usepackage{fancyhdr}
\usepackage{lastpage}
\usepackage{hyperref}

\usepackage{float}

\usepackage{graphicx}
\graphicspath{{./img/}}
\usepackage{tikz}

% Reset paragraph indentation -------------------------------------------------
\setlength{\parindent}{0cm}

% Allow a paragraph to have a linebreak ---------------------------------------
\newcommand{\paragraphnl}[1]{\paragraph{#1}\mbox{}\\}

% Page header and footer ------------------------------------------------------
\pagestyle{fancy}
\setlength{\headheight}{33pt}
\renewcommand{\headrulewidth}{0.5pt}
\lhead{\includegraphics[height=1cm]{hepia.jpg}}
\chead{Jackpot}
\rhead{Claudio Sousa - David Gonzalez}
\renewcommand{\footrulewidth}{0.5pt}
\lfoot{1 février 2017}
\cfoot{}
\rfoot{Page \thepage /\pageref{LastPage}}

% Table of contents depth -----------------------------------------------------
\setcounter{tocdepth}{3}

% Document --------------------------------------------------------------------
\begin{document}

\title
{
    \Huge{Programmation concurrente} \\
    \Huge{Jackpot}
}
\author
{
    \LARGE{David Gonzalez - Claudio Sousa}
}
\date{1 février 2017}
\maketitle

\begin{center}
    %\includegraphics[scale=0.27]{logo.png}
\end{center}

\thispagestyle{empty}

\newpage

% -----------------------------------------------------------------------------
\section{Introduction}

Ce TP de deuxième année consiste à implémenter une machine à sous de Casino multi-tâches.

\subsection{Spécification fonctionnelle}

Chaque partie débute avec l'insertion d'une pièce.
Ensuite, 3 roues tournent à des vitesses différentes.
Chaque roue est arrêtée consécutivement (de gauche à droite)
soit manuellement par l'utilisateur, soit automatiquement après 3 secondes. \\

Lorsque toutes les roues sont arrêtées, le résultat est ajouté à la caisse de la machine.
Ce résultat est calculé selon le nombre de chiffres identiques, les résultats possibles sont:
\begin{itemize}
    \item aucun chiffre identique (perdu);
    \item 2 chiffres identiques (2 pièces gagnées);
    \item 3 chiffres identiques (moitié des pièces en caisse). \\
\end{itemize}

Ce résultat est affiché pendant 5 secondes, puis revient à l'insertion de la pièce.

\subsubsection{Entrées}

Les entrée sont gérées à l'aide de signaux générés par la console avec les touches suivantes:
\begin{itemize}
    \item CTRL-Z (SIGTSTP): insertion d'une pièce;
    \item CTRL-C (SIGINT): arrêt d'une roue manuellement;
    \item CTRL-\textbackslash{} (SIGQUIT): arrêt du jeu.
\end{itemize}

\subsubsection{Threads}

Les threads sont divisés ainsi:
\begin{itemize}
    \item 1 thread pour le controlleur du jeu (qui sera le seul à recevoir les signaux);
    \item 1 thread pour l'affichage;
    \item N threads, 1 pour chaque roue.
\end{itemize}

\newpage

% -----------------------------------------------------------------------------
\section{Development}
\subsection{Architecture}

\begin{figure}[H]
    \begin{center}
        \includegraphics[width=\textwidth]{modules.png}
    \end{center}
    \caption{Architecture du Jackpot}
    \label{Architecture du Jackpot}
\end{figure}

\subsubsection{game\_t et game\_data\_t}
Ces deux structures représentent toutes les données partagées entre les 3 modules pricipaux,
qui sont: \textit{Controller}, \textit{Wheel}, \textit{Display}. \\

La première (\textit{game\_t}) contient les données d'état des roues ainsi que les données de synchronisation. \\

La deuxième (\textit{game\_data\_t}) contient les données du jeu en tant que telle:
la valeur de chaque roue, l'argent restant et l'état final du jeu (gagné, perdu).

\subsubsection{Main}
La fonction principale du programme ne fait que lancer le \textit{Controller}.

\newpage

\subsubsection{Controller}
Ce module est le coeur du programme.
Il est responsable d'instancier le module \textit{Display} ainsi que toutes les instance de \textit{Wheel} (3 par défaut).
Il est également chargé de contrôler l'avancement du jeu selon la machine d'état ci-dessous:

\begin{figure}[H]
    \begin{center}
        \includegraphics[width=0.8\textwidth]{machinestate.png}
    \end{center}
    \caption{Machine d'état du contrôleur}
    \label{Machine d'état du contrôleur}
\end{figure}

Comme le dit la spécification, les entrée utilisateurs sont gérées par des signaux.
Le \textit{Controller} est donc le seul module (et \textit{thread}) à recevoir et à traiter ces signaux. \\

Les communications sont faitent aux travers des deux structures citées (\textit{game\_t} et \textit{game\_data\_t}),
qu'il est chargé d'instantier et d'initialiser. \\

Au niveau des \textit{threads}, le \textit{Controller} crée lui-même son propre \textit{thread} durant son instanciation.

\subsubsection{Wheel}
Le module \textit{Wheel} est responsable de faire tourner 1 roue (changer le chiffre) dans un \textit{thread} séparé à une certaine fréquence.

\subsubsection{Display}
Le module \textit{Display} a le rôle de gérer l'affichage du jeu dans son \textit{thread}. Donc:
\begin{itemize}
    \item le message de début de parti;
    \item les roues lorsque la parti est en cours;
    \item le résultat de cette parti;
    \item le message lorsqu'on quitte le jeu.
\end{itemize}

\subsubsection{Timer}
Ce module contient deux fonctions utilitaires permettant de mesurer le temps et
d'attendre à une certaine fréquence.

\newpage

% -----------------------------------------------------------------------------
\subsection{Flux d'exécution et synchronisation}

\begin{figure}[H]
    \begin{center}
        \includegraphics[width=0.95\textwidth]{flow.png}
    \end{center}
    \caption{Flux d'exécution et synchronisation}
    \label{Flux d'exécution et synchronisation}
\end{figure}

\newpage

% -----------------------------------------------------------------------------
\subsection{Méthodologie de travail}
\subsubsection{Répartition du travail}

Ce travail a été effectué à deux.

Nous avons commencés par réfléchir sur papier sur deux éléments:

\begin{itemize}
    \item architecture du programme: modules et interfaces de bases;
    \item premier jet de synchronisation entre les différents threads. \\
\end{itemize}

Ensuite, le travail a été réparti ainsi:
\begin{itemize}
    \item ... \\
\end{itemize}

Finalement, nous avons mis en commun les modules et finalisés la synchronisation des threads des différents modules.

\newpage

\end{document}
